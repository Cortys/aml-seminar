% !TEX root = ../main.tex
% chktex-file 21
\section{Optimizing Training}%
\label{sec:params}

Now follows an overview of approaches to speed up the training process.
We will discuss four approaches:
\begin{enumerate}
	\item A general purpose method that combines subsampling with bootstrapping.
	\item An iterative method to select the optimal subsample size during gradient descent.
	\item Improving the quality of subsampling for logistic regression by weighing the samples.
	\item Speeding up the training of SVMs via \(k\)-means clustering.
\end{enumerate}

\subsection{Bag of Little Bootstraps}%
\label{sec:params:blb}

The first approach we will discuss is called \textit{Bag of Little Bootstraps}~\cite{Kleiner2011}.
It combines subsampling with bootstrapping and is particularly well suited for parallelized implementations.

\textit{TODO:\@ explain BLB}

\subsection{Subsample Size Selection for Gradient Descent}%
\label{sec:params:samplesize}

Next we will discuss gradient descent based optimization.
The sample size \(S\) that describes the number of datapoints that are considered in a single gradient descent step heavily influences the optimizer's behavior:
\begin{itemize}
	\item In the stochastic approximation regime small samples, typically \(S = 1\), are used. This causes fast but noisy steps.
	\item In the batch regime large samples, typically \(S = |\Dtrain|\), are used. Steps are expensive to compute but more reliable.
\end{itemize}
Both extremes are usually not suitable for Big Data applications.
Very small samples cannot be parallelized well, making them a bad fit for the compute clusters that are typically available nowadays.
The gradients for very big samples however are often too slow to compute.
\(S\) should ideally lie somewhere in between.

\citet{Byrd2012} describe an iterative algorithm that dynamically increases \(S\) as long as this promises to significantly reduce the gradient noise.

\subsection{Subsampling for Logistic Regression}%
\label{sec:params:osmac}

Subsampling usually increases the mean squared error (MSE) of the resulting hypothesis compared to one that is trained on the full dataset.
OSMAC~\cite{Wang2017} is a method that improves upon na{\"\i}ve subsampling by weighing the samples.

\textit{TODO:\@ explain OSMAC}

\subsection{SVM-KM}%
\label{sec:params:svmkm}

To speed up the training of SVMs \citet{Almeida2000} proposed a simple method that reduces the dataset size via \(k\)-means clustering.
It can be described as a three-step procedure:
\begin{enumerate}
	\item Group the training samples \(\Dtrain\) into \(k\) clusters \(C_1, \dots, C_k\) with centers \(c_1, \dots, c_k\) where \(k\) should be determined via hyperparameter optimization.
	\item Check for each cluster \(C_i\) whether all associated datapoints belong to the same class, i.~e.\@ \(\exists\, z \in \{+1, -1\}: \forall (x, y) \in C_i: y = z\).
		If yes, all datapoints in \(C_i\) are removed from \(\Dtrain\) and replaced by \(c_i\).
		If not, they are kept in the dataset.
		The intuition behind this is that clusters with points from multiple classes might be near the decision boundary so they are kept to serve as potential support vectors.
	\item Finally standard SVM training is performed on the reduced training dataset.
\end{enumerate}

\subsubsection{Evaluation}%
\label{sec:params:svmkm:eval}

\textit{TODO}
